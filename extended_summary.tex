\documentclass[a4paper,11pt]{article}

\usepackage[english]{babel}
\usepackage[utf8]{inputenc}
\usepackage{graphicx}
\usepackage[T1]{fontenc}
\usepackage{setspace}
\usepackage{geometry}
\usepackage{titling}
\usepackage{float}
\usepackage{lipsum}
\usepackage[compact]{titlesec}
\titlespacing*{\title}{0pt}{*1}{*1}
%\usepackage{savetrees}
 \geometry{
 a4paper,
 total={170mm,257mm},
 left=20mm,
 top=20mm,
 right=20mm,
 bottom=20mm,
 }
\title{\normalsize\textbf{Fault Tolerance Capabilities of Three, Four and Six Phase Configurations of a 24 Slot Modular PMSM}}
\date{}
\titlespacing\section{0pt}{11pt plus 4pt minus 2pt}{0pt plus 2pt minus 2pt}
\titlespacing\subsection{0pt}{11pt plus 4pt minus 2pt}{0pt plus 2pt minus 2pt}
\titlespacing\subsubsection{0pt}{11pt plus 4pt minus 2pt}{0pt plus 2pt minus 2pt}
\begin{document}
\vspace{-45mm}
\maketitle
\vspace{-30mm}
\textit{\normalsize\textbf{Abstract:}}
\textit{Integrated Modular Motor Drive (IMMD) is the concept of integrating power electronics, electric motor and other passive components of a drive to obtain a more compact system compared to conventional motor drives. In this study, fault tolerance and redundancy capabilities of different phase and winding configurations of an IMMD system is investigated. This is possible by manipulating gate drive signals of the inverter and end winding connections, without making any changes on the stator side of the electric machine. Three, four, symmetric and asymmetric six-phase topologies are described. Control strategies and redundancy possibilities of these different topologies under an open circuit fault condition are examined in MATLAB/Simulink environment and validated with Finite Element Analysis (FEA) software ANSYS/Maxwell. }

\section{\normalsize\textbf{Introduction}}
With the developments in safety-critical industrial applications, fault tolerance, reliability and redundancy of electric motor drive systems gain popularity

\section{\normalsize\textbf{Design Specifications of IMMD System}}

 
\section{\normalsize\textbf{Three, Four and Six-Phase Winding Configurations}}
Burada değişik konfigürasyonları açıklıyoruz. Görsel olabilir. Winding factor, drive tarafında yapılması gereken değişiklikler (sürüş yöntemi falan) açıklanabilir.

\section{\normalsize\textbf{Comparison of Different Topologies}}
Redundancy yüzdesi, normal operation, one phase open fault durumunda olacaklar, ilk IMMD topolojisiyle kıyaslanması, fault durumunda akımların, torque ripple ve average torque'un değişimi belirtilip karşılaştırılacak. Tablolanabilir. Akım vs gibi şeylerin simülasyonu genel olarak Simulink üzerinden yürüyecek. Tork için FEA karşılaştırması da eklenecek. Fault durumunda düzgün operate ettirmek için faz açısı kaydırma mevzularına girilecek. Genel olarak karşılaştırma bu değşik modül ve faz sayılı topolojilerin redundancy karşılaştırması üzerinden yürüyecek. 

\section{\normalsize\textbf{Conclusions}}
Toparlama. Ne yapmıştık bugün. Ne kattık biz bu çalışmayla. Experimental sonuç verecek miyiz full paperda, neleri ekleriz full papera onları yazarız.

\section{\normalsize\textbf{References}}
\#plagiarismehayır

\end{document}
