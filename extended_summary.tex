\documentclass[a4paper,11pt]{article}

\usepackage[english]{babel}
\usepackage[utf8]{inputenc}
\usepackage{graphicx}
\usepackage[T1]{fontenc}
\usepackage{setspace}
\usepackage{geometry}
\usepackage{titling}
\usepackage{float}
\usepackage{lipsum}
\usepackage[compact]{titlesec}
\titlespacing*{\title}{0pt}{*1}{*1}
%\usepackage{savetrees}
 \geometry{
 a4paper,
 total={170mm,257mm},
 left=20mm,
 top=20mm,
 right=20mm,
 bottom=20mm,
 }
\title{\normalsize\textbf{Fault Tolerance Capabilities of Three, Four and Six Phase Configurations of a 24 Slot Modular PMSM}}
\date{}
\titlespacing\section{0pt}{11pt plus 4pt minus 2pt}{0pt plus 2pt minus 2pt}
\titlespacing\subsection{0pt}{11pt plus 4pt minus 2pt}{0pt plus 2pt minus 2pt}
\titlespacing\subsubsection{0pt}{11pt plus 4pt minus 2pt}{0pt plus 2pt minus 2pt}
\begin{document}
\vspace{-45mm}
\maketitle
\vspace{-30mm}
\textit{\normalsize\textbf{Abstract:}}
\textit{In this paper bla bla bla}
\section{\normalsize\textbf{Introduction}}
Burada kısaca bizim bu çalışmayı yapmamızdaki motivasyonu falan özetliyoruz. Giriş işte.

\section{\normalsize\textbf{Design Specifications of IMMD System}}
Burada kısaca IMMD'nin tasarımını özetliyoruz. Boyut, pole-slot sayısı, rated değerler gibi şeyleri tablo halinde veriyoruz. Tasarım sürecinde aldığımız kararları ve nedenlerini yazıyoruz. Çok uzun tutmaya gerek yok, full paperda detayına girilir. Extended summary için introyla birlikte burası toplam 1 sayfa falan olacak.
 
\section{\normalsize\textbf{Three, Four and Six-Phase Winding Configurations}}
Burada değişik konfigürasyonları açıklıyoruz. Görsel olabilir. Winding factor, drive tarafında yapılması gereken değişiklikler (sürüş yöntemi falan) açıklanabilir.

\section{\normalsize\textbf{Comparison of Different Topologies}}
Redundancy yüzdesi, normal operation, one phase open fault durumunda olacaklar, ilk IMMD topolojisiyle kıyaslanması, fault durumunda akımların, torque ripple ve average torque'un değişimi belirtilip karşılaştırılacak. Tablolanabilir. Akım vs gibi şeylerin simülasyonu genel olarak Simulink üzerinden yürüyecek. Tork için FEA karşılaştırması da eklenecek. Fault durumunda düzgün operate ettirmek için faz açısı kaydırma mevzularına girilecek. Genel olarak karşılaştırma bu değşik modül ve faz sayılı topolojilerin redundancy karşılaştırması üzerinden yürüyecek. 

\section{\normalsize\textbf{Conclusions}}
Toparlama. Ne yapmıştık bugün. Ne kattık biz bu çalışmayla. Experimental sonuç verecek miyiz full paperda, neleri ekleriz full papera onları yazarız.

\section{\normalsize\textbf{References}}
\#plagiarismehayır

\end{document}
